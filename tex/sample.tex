\documentclass[a4j,10pt]{jsarticle}
\usepackage{mylatex}
\usepackage{ap3}
\usepackage{ascmac}

\title{応用プログラミング3 提出課題1}
\author{C0110000 井上亮文}
\date{2012年4月19日(木)}

\begin{document}
\maketitle


\section{目的}

今回の課題の目的を述べる.


\section{課題1}

\subsection{問題}

Servlet と JSP を連携させて,アクセス時の日付と現在時刻を表示するウェブア
プリケーションを作成しなさい.日付と時刻は Servlet 内で計算すること.


\subsection{ソースコード}

% \lstinputlisting[caption=HelloWorld.java,label=pg:k1-1s]{src/HelloWorld.java}

% \pgref{pg:k1-1s}のソース15行目で {\tt Hello, }という文字列を出力している.
% 次の16行目では{\tt Bye-bye 応用プログラミング!}という文字列を出力してい
% る.

\subsection{ファイル構成}

Webアプリケーションは複数ファイルをいくつかのディレクトリに分けて保存す
ることが多い.その構成を図表で書いておくと状況が整理しやすいのでオススメ.


\subsection{実行結果}

% \lstinputlisting[caption=実行結果2-1,label=pg:k1-1r]{src/k1-1r.txt}


\subsection{考察}

% \pgref{pg:k1-1r} より,本課題の最低限の仕様を満たすことができたと言える.
しかし,見栄えや再利用性を考えた場合,あんなことやこんなことが必要になる
と考えられる.

…みたいに,実行結果を踏まえて現状を改善するにはどうすればいいか等を書く
と格好がつく.

\section{課題2}

複数の課題が出た場合はセクションを分けて書くこと.


\end{document}
